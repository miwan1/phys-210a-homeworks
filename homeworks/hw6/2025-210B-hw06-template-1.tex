\documentclass[12pt]{report}

\newcommand\htmladdnormallink[2]{\href{#2}{#1}}

\textheight 22cm
\textwidth 15.5cm
\oddsidemargin 0pt\evensidemargin 0pt
%\oddsidemargin 14pt\evensidemargin 0pt
%\topmargin -40pt
\topmargin-30pt
%\bottommargin0pt
\def\baselinestretch{1.1}
%\addtolength{\parskip}{1ex}
\jot=.5ex
%\parskip = 0.02in


\setlength\arraycolsep{2pt}



\usepackage{amssymb}
\usepackage{amsmath,bm}
\usepackage{amssymb}
\usepackage{graphicx}
\usepackage{amsfonts}         
\usepackage{fancybox}   

%\usepackage[numbers]{natbib}

\usepackage{enumitem}

\usepackage{slashed}




\usepackage[usenames,dvipsnames]{xcolor}%http://en.wikibooks.org/wiki/LaTeX/Colors

\definecolor{darkgreen}{rgb}{0,0.4,0}
\definecolor{darkred}{rgb}{0.4,0,0}
\definecolor{darkblue}{rgb}{0,0,0.4}
\definecolor{lightblue}{rgb}{.6,.6,0.9}
\newcommand{\cobl}{\color{darkblue}}

\newcommand{\cor}{\color{red}}
\newcommand{\cog}{\color{darkgreen}}
\newcommand{\cob}{\color{black}}

\definecolor{uglybrown}{rgb}{0.8,  0.7,  0.5}

\def\ii{{\bf i}}
\def\Ione{\mathbb{I}}
\def\UU{{\bf U}}
\def\HH{{\bf H}}
\def\pp{{\bf p}}
\def\aa{{\bf a}}
\def\qq{{\bf q}}
\def\eps{\epsilon}
\def\half{{1\over 2}}
\def\Tr{{{\rm Tr~ }}}
\def\tr{{\rm tr}}
\def\Re{{\rm Re\hskip0.1em}}
\def\Im{{\rm Im\hskip0.1em}}
\def\ppi{\boldsymbol{\pi}}
\def\pphi{\boldsymbol{\phi}}
%\def\pphi{\phi}
\def\grad{\vec \nabla}
\def\vB{\vec B}
\def\vE{\vec E}
\def\vA{\vec A}
\def\vAA{ \vec{\bf A}}
\def\vEE{{{\vec {\bf E}}}}
\def\vBB{{\vec {\bf B}}}

\def\CL{{\cal L}}



\def\bra#1{\left\langle#1\right|}
\def\ket#1{\left|#1\right\rangle}
\def\bbra#1{{\langle\langle}#1|}
\def\kket#1{|#1\rangle\rangle}
\def\vev#1{\left\langle{#1}\right\rangle}

\def\ketbra#1#2{ | #1 \rangle\hskip-2pt\langle #2|}



\def\be{\begin{equation}}
\def\ee{\end{equation}}
\def\({\left(}
\def\){\right)}



\newcommand{\bea}{\begin{eqnarray}}
\newcommand{\eea}{\end{eqnarray}}


\def\parfig#1#2{
\parbox{#1\textwidth}
{\includegraphics[width=#1\textwidth]{#2}}
}



%%from  Rudro Rana Biswas
\usepackage[pagebackref,  % this puts links to the page numbers where refs appear
%pdftex, 
bookmarks={false}, pdfauthor={John McGreevy}, pdftitle={Yay, physics!}]{hyperref}
\hypersetup{colorlinks=true, 
linkcolor=BrickRed, 
citecolor=Violet, 
filecolor=OliveGreen, 
urlcolor=RoyalBlue, 
filebordercolor={.8 .8 1}, 
urlbordercolor={.8 .8 0}}%http://en.wikibooks.org/wiki/LaTeX/Hyperlinks


%\usepackage{mathtools} % for inclusion arrow \xhookrightarrow{}


\renewcommand{\theequation}{\arabic{equation}}
\newif{\ifeq}           % defines a new condition @eq tested by the conditional \ifeq
\eqtrue                 % if uncommented, declares @eq to be true
%\eqfalse              % if uncommented, declares @eq to be false
%                                %
%                                % to use this, wrap text with the conditional, eg:
%                                %
%                                % \ifeq
%                                % SHOW THIS IFF \eqtrue HAS BEEN DECLARED
%                                % \fi
%
\def\answer#1
{
\ifeq
\textcolor{darkblue}{#1}
\fi
}

\begin{document}
\begin{center}

University of California at San Diego -- 
Department of Physics --
Prof.~John McGreevy

{\Large\bf  Physics 210B Non-equilibrium  Fall 2025}\\
{\Large\bf Assignment 6 \answer{--~~~ Solutions} }
\end{center}


\noindent
\hfill {\bf Due 11:59pm {Monday, November 10, 2025}} 




%I will add another problem later today.
\bigskip
\hrule




\begin{enumerate}
\item {\bf Positivity and Fokker-Planck.}
Consider the Fokker-Planck equation for the evolution of the probability density $P(x,t) \equiv P(x,t|0,0)$
in 1d,
\be \partial_t P(x,t) = D \partial_x \( \partial_x P + {1\over T} U' P \)
\equiv - \hat L P.\ee
(I used the fluctuation-dissipation relation to eliminate the friction coefficient $\mu = D/T$.)
Let $ P(x,t) \equiv \sqrt{P_\text{eq}(x)} Q(x,t) $, 
where $P_\text{eq}(x) \equiv e^{ - U(x)/T}$ is the equilibrium distribution.

\begin{enumerate}
\item Show that the FP equation can be written as $\partial_t Q = - A^\dagger A Q $ for some operator $A$.
Since $A^\dagger A \geq 0$, we see that all the eigenvalues of $\hat L$ must be non-negative.

\answer{
    \be \partial_t P = e^{ - U/2T } \partial_t Q \ee
    \be e^{-U/2T} \partial_t Q = D \partial_x \( \partial_x \( e^{ - U/2T } Q \) + {1\over T} U' e^{ - U/2T } Q \) \ee
    \be \partial_x(e^{-U/2T}Q)+ {1\over T} U' e^{-U/2T}Q = e^{-U/2T} ( \partial_x Q - U'/2T Q + U'/T Q ) = e^{-U/2T} ( \partial_x Q + U'/2T Q ) \ee
    \be D \partial_x \( e^{-U/2T} ( \partial_x Q + U'/2T Q ) \) = D e^{-U/2T} ( \partial_x^2 Q + U'/T \partial_x Q + ( U''/2T - (U')^2/4T^2 ) Q ) \ee
    This is equal to
    \be De^{-U/2T} ( \partial_x - U'/2T ) ( \partial_x + U'/2T ) Q \ee
    Therefore, 
    \be \partial_t Q = D ( \partial_x - U'/2T ) ( \partial_x + U'/2T ) Q \ee
    \be \partial_t Q = - A^\dagger A Q \ee
    where
    \be A = \sqrt{D} ( \partial_x + U'/2T ) , ~~~ A^\dagger = \sqrt{D} ( - \partial_x + U'/2T ) . \ee 
}


\item Show that if $U(x) = \kappa x^2 /2 $ is quadratic (the Ornstein-Uhlenbeck process), then $A^\dagger$ and $A$ are the usual raising and lowering operators for a quantum mechanical harmonic oscillator.
Find the steady-state distribution and work out the spectrum of relaxation times in this case.

\answer{
    \be U(x) = \kappa x^2 /2  \implies U'(x) = \kappa x \ee
    \be U'/2T = \kappa x / 2T \implies U'/2T = \alpha x\ee where $ \alpha \equiv \kappa / 2T $.\\
    We can see that $A$ and $A^\dagger$ are the raising and lowering operators of a harmonic oscillator.
    \be A = \sqrt{D} ( \partial_x + \alpha x ) , ~~~ A^\dagger = \sqrt{D} ( - \partial_x + \alpha x ) . \ee
    Steady state:
    \be AQ_0 = 0 \implies ( \partial_x + \alpha x ) Q_0 = 0 \implies Q_0(x) = C e^{ - \alpha x^2 /2 } \ee
    \be C = \( {\alpha \over 2 \pi } \)^{1/4} \text{ from normalization } \ee
    \be \left[A^\dagger, A \right] = \frac{D\kappa}{T} = \gamma \ee
    So the FP equation for each mode is
    \be \partial_t Q_n = -(A^\dagger A)Q_n = -n\gamma Q_n \ee
    \be Q_n(x,t) = Q_n(x,0) e^{-n\gamma t} \ee
    Thus the relaxation times are
    \be \tau_n = {1 \over n \gamma } = { T \over n D \kappa } \ee
    where n = 0, 1, 2, ...
}

\end{enumerate}


\item {\bf Coarse-graining.} 
 Why can we describe a complicated reaction in terms of just one coordinate $x$ (as we do, for example, in the next problem)?   
Suppose that there are really many variables, not just $x$ but also $y_1, y_2 \cdots y_N$.  Then the joint overdamped Langevin equations read 
\be\label{eq:langevin-xy} \mu \dot x = - \partial_x U(x, y) + \xi(t) , ~~~
\mu_i \dot y_i = - \partial_{y_i} U(x,y) + \xi_i \ee
with $ \vev{ \xi_i(t) \xi_j(t') } = 2 T \mu_i \delta_{ij} \delta(t-t')$ satisfying the fluctuation-dissipation relation.  

Imagine that the dynamics of $x$ is slow compared to that of the $y$ variables, so we can do a Born-Oppenheimer approximation and integrate out the $y$ variables at fixed $x$.  

\begin{enumerate}

\item Verify that, under the Langevin evolution \eqref{eq:langevin-xy}, the stationary distribution of the $y_j$ at fixed $x$ is the Boltzmann distribution: 
\be P( y | x ) = {1\over Z(x)} e^{ - U(x,y)/T } . \ee

\answer{
    Because the dynamics of y is slow compare to that of x, we can have:
    \be \dot y_i = - {1 \over \mu_i} \partial_{y_i} U(x,y) + \eta_i(t) \ee
    where $\eta_i(t) = {1 \over \mu_i} \xi_i(t)$ with $ \vev{ \eta_i(t) \eta_j(t') } = {2 T \over \mu_i} \delta_{ij} \delta(t-t')$.
    Therefore, with fixed x, we have an overdamped langevin eq for each $y_i$ with:
    \be A_i = -1/\mu_i \partial_{y_i} U(x,y) , ~~~ D_i = { T \over \mu_i } \ee
    Suppose that $P(y,t|x)$ is the probability density of y at time t with fixed x, then the Fokker-Planck equation is:
    \be \partial_t P = -\sum_{i=1}^{N} \partial_{y_i} ( {1\over \mu_i} \partial_{yi} U(x,y) P ) + \sum_{i=1}^{N} {T \over \mu_i} \partial_{y_i}^2 P \ee
    This can be written as:
    \be \partial_t P = -\sum_{i} \partial_{y_i} J_i \ee
    where
    \be J_i = - {1\over \mu_i} \partial_{y_i} U(x,y) P - {T \over \mu_i} \partial_{y_i} P \ee
    At steady state, $\partial_t P = 0$, so
    \be \sum_{i} \partial_{y_i} J_i = 0 \ee
    A sufficient condition for this is $J_i = 0$ for all i, which gives:
    \be - {1\over \mu_i} \partial_{y_i} U(x,y) P - {T \over \mu_i} \partial_{y_i} P = 0 \ee
    \be \partial_{y_i} P = - {1 \over T} \partial_{y_i} U(x,y) P \ee
    \be P(y|x) = C(x) e^{ - U(x,y)/T } \ee
    where $C(x)$ is a normalization constant $\equiv {1 \over Z(x)}$.
}

\item Suppose that the $y$ variables equilibrate quickly, and the $x$ variable moves in their equilibrium distribution.  
So let's average the dynamics of $x$ over the stationary distribution $P(y|x)$.  Show that this generates an equation where $x$ moves in an effective potential
\be \mu \dot x = - \partial_x U_\text{eff}(x) + \xi(t)  \ee
where 
\be U_\text{eff}(x) = F(x) = - T \log Z(x)  \ee
is the free energy at fixed $x$.
Check that this gives the right answer when $x$ is decoupled from $y$:  $U(x,y) = u(x) + v(y)$.  

\answer{
    From the original Langevin equation for x:
    \be \mu \dot x = - \partial_x U(x, y) + \xi(t) \ee
    Taking the average over the stationary distribution of y at fixed x, we have:
    \be \mu \dot x = - \int dy P(y|x) \partial_x U(x, y) + \xi(t) \ee
    \be = - \partial_x \int dy P(y|x) U(x, y) + \xi(t) \ee
    Using $P(y|x) = {1\over Z(x)} e^{ - U(x,y)/T }$, we get:
    \be = - \partial_x \( {1\over Z(x)} \int dy e^{ - U(x,y)/T } U(x, y) \) + \xi(t) \ee
    Note that:
    \be F(x) = - T \log Z(x) = - T \log \( \int dy e^{ - U(x,y)/T } \) \ee
    Therefore,
    \be U_\text{eff}(x) = F(x) = - T \log Z(x)  \ee
    So we have:
    \be \mu \dot x = - \partial_x U_\text{eff}(x) + \xi(t)  \ee
    In the case where $U(x,y) = u(x) + v(y)$, we have:
    \be Z(x) = e^{-u(x)/T} \int dy e^{-v(y)/T} = e^{-u(x)/T} Z_y  \ee
    where $Z_y$ is a constant independent of x.
    Thus,
    \be F(x) = - T \log Z(x) = u(x) - T\log Z_y  \ee
    The effective potential is just $u(x)$ up to an additive constant, which does not affect the dynamics.
}

\item{} [Bonus problem] The equation \eqref{eq:langevin-xy} is not completely  general.
We could have written 
\be \gamma_{IJ} \dot x_J = - \partial_{x^I} U + \xi_I \ee
where $\{x^I\}  = \{ x, y^i\} $.
The basis in which the noise correlations is diagonal $ \vev{\xi^I(t) \xi^J(t') } = \delta^{IJ} \delta(t-t') $
may not be the basis in which the damping matrix $\gamma_{IJ}$ is diagonal, as it was above.
Does this change the story?

\end{enumerate}

\item {\bf Numerical experiments on activation over a barrier.}  
Consider a 1d particle with a (symmetric) double-well potential 
$U(x) = U_0 \( 1 - \( {x \over x_0}\)^2 \) ^2 $, with minima at $x=\pm x_0$ and a barrier of height $U_0$ in between.
We want to simulate the dynamics of the overdamped Langevin equation
\be \mu \dot x(t) = - \partial_x U(x) + \xi(t)  \ee
where $\xi$ is Gaussian white noise with covariance $ \vev {\xi(t) \xi(t') } = 2 \mu T \delta(t-t') $. 

\begin{enumerate}
\item 
Convince yourself or me (whichever is harder) that the Euler method 
(integrating the Langevin equation from $t_n = n \Delta t$ to $t_{n+1} = t_n + \Delta t$ 
and replacing $\dot x$ with $ x_{n+1} - x_n \over \Delta t $)
gives the following discretization
\be y_{n+1} = y_n + \alpha E y_n \( 1 - y_n^2 \) + \sqrt{ \alpha \over 2 } \eta_n \ee
where $y\equiv x/x_0$, 
$ \alpha = {4 T \Delta t \over \gamma x_0^2 } $ is a small number, 
$E \equiv U_0 /T $ is the normalized activation energy for escape over the barrier,
and $\eta_n = {\tt randn}()$ is an independent, zero-mean, unit-variance, Gaussian random variable for each time step $n$.


\answer{
I was not sure how to get to the discretization but will take this as given for now to simulate part (b-d). 
}

\item Implement this evolution rule numerically.  



\item Check numerically that the distribution of $x$ is given by the Boltzmann distribution $P(x) \propto e^{ - U(x)/T}$. 

\answer{
    From Figure 1, we can see that the histogram of x from simulation matches well with the plotted Boltzmann distribution.
    \begin{figure}[h]
        \centering
        \includegraphics[width=0.6\textwidth]{part_b_c.png}
        \caption{Histogram of x from simulation vs plotted Boltzmann distribution}
    \end{figure}
}

\item Start at a small value of $E$ and gradually increase it.  
At some point you should see isolated discrete events corresponding to the reaction happening, i.e.~the particle jumping from one well to the other.  Use the simulation to estimate the rate of these jumps and plot it as a function of $E$.  Does it give the Arrhenius law?
\answer{
    \begin{figure}[h]
        \centering
        \includegraphics[width=0.6\textwidth]{part_d.png}
        \caption{Jump rate vs E from simulation}
    \end{figure}
    From Figure 2, we can see that the jump rate decreases as E increases, roughly following an exponential decay trend. This is consistent with the Arrhenius law in lecture notes eq. 2.144. 
}
\end{enumerate}







\item{\bf Time-reversal symmetry of the MSR action and the fluctuation-dissipation relation.}
[Bonus problem]
Consider the MSR action for an overdamped Brownian particle in a potential $U(x)$:
\be S_\text{MSR}[x, \tilde x] = \int dt \( D \tilde x^2+ \ii \tilde x \dot x - \ii \tilde x \upsilon(x) \) \ee
where $\upsilon(x) = - \mu \partial_x U $.  

\begin{enumerate}

\item 
Consider the transformation
\be x(t) \mapsto x'(t) \equiv x(s), \tilde x(t) \mapsto \tilde x'(t) \equiv \tilde x(s) + {\ii \over D} \dot x(s) ,
\text{ with } s\equiv - t ~.\ee
Note that under this transformation $\dot x(t) \mapsto - \dot x(s)$.  
Show that the action is invariant, 
\be S_\text{MSR}[x, \tilde x] = S_\text{MSR}[x', \tilde x'] \ee
up to boundary terms.


\item Show that this transformation is involutive, that is, if you do it twice you get back the original configuration.  



\item Convince yourself that this invariance of the action implies the Ward identity
\be Z[J, \tilde J] = Z[J', \tilde J']  \label{eq:ward}\ee
where 
\be Z[J, \tilde J] \equiv \int Dx D\tilde x e^{ - S_\text{MSR}[x, \tilde x] + \int dt \( J x + \tilde J \tilde x\)  } ~
\ee
and 
\be J'(t) \equiv J(-t) - {1\over D} \partial_t \tilde J(-t) , ~~~\tilde J'(t) = \tilde J(-t ) . \ee
Note that this requires checking that the Jacobian of the transformation is 1.  



\item Show that the Ward identity \eqref{eq:ward} implies the fluctuation dissipation theorem, in the form
\be \ii \chi(t,t') = - {1\over D} \partial_{t'} C(t,t') \ee
for $t > t'$, where 
\be C(t,t') \equiv \vev{x(t) x(t') } = { \delta \over \delta J(t) }  { \delta \over \delta  J(t') }  Z |_{J=\tilde J=0} \ee
is the correlation function 
and 
\be \ii  \chi(t,t') \equiv \ii  { \delta \vev{x(t)}_f \over \delta f(t') } |_{f=0} =
\vev{x(t) \tilde x(t') } = { \delta \over \delta J(t) }  { \delta \over \delta \tilde J(t') }  Z |_{J=\tilde J=0}
 \ee
is the response function.

[A paper that uses this logic to generalize the FDT away from equilibrium  is 
\htmladdnormallink{this one}{https://arxiv.org/abs/1007.5059}.]

\end{enumerate}



\end{enumerate}
\end{document}

