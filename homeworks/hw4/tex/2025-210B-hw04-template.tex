\documentclass[12pt]{report}

\newcommand\htmladdnormallink[2]{\href{#2}{#1}}

\textheight 22cm
\textwidth 15.5cm
\oddsidemargin 0pt\evensidemargin 0pt
%\oddsidemargin 14pt\evensidemargin 0pt
%\topmargin -40pt
\topmargin-30pt
%\bottommargin0pt
\def\baselinestretch{1.1}
%\addtolength{\parskip}{1ex}
\jot=.5ex
%\parskip = 0.02in


\setlength\arraycolsep{2pt}



\usepackage{amssymb}
\usepackage{amsmath,bm}
\usepackage{amssymb}
\usepackage{graphicx}
\usepackage{amsfonts}         
\usepackage{fancybox}   

%\usepackage[numbers]{natbib}

\usepackage{enumitem}

\usepackage{slashed}




\usepackage[usenames,dvipsnames]{xcolor}%http://en.wikibooks.org/wiki/LaTeX/Colors

\definecolor{darkgreen}{rgb}{0,0.4,0}
\definecolor{darkred}{rgb}{0.4,0,0}
\definecolor{darkblue}{rgb}{0,0,0.4}
\definecolor{lightblue}{rgb}{.6,.6,0.9}
\newcommand{\cobl}{\color{darkblue}}

\newcommand{\cor}{\color{red}}
\newcommand{\cog}{\color{darkgreen}}
\newcommand{\cob}{\color{black}}

\definecolor{uglybrown}{rgb}{0.8,  0.7,  0.5}

\def\ii{{\bf i}}
\def\Ione{\mathbb{I}}
\def\UU{{\bf U}}
\def\HH{{\bf H}}
\def\pp{{\bf p}}
\def\aa{{\bf a}}
\def\qq{{\bf q}}
\def\eps{\epsilon}
\def\half{{1\over 2}}
\def\Tr{{{\rm Tr~ }}}
\def\tr{{\rm tr}}
\def\Re{{\rm Re\hskip0.1em}}
\def\Im{{\rm Im\hskip0.1em}}
\def\ppi{\boldsymbol{\pi}}
\def\pphi{\boldsymbol{\phi}}
%\def\pphi{\phi}
\def\grad{\vec \nabla}
\def\vB{\vec B}
\def\vE{\vec E}
\def\vA{\vec A}
\def\vAA{ \vec{\bf A}}
\def\vEE{{{\vec {\bf E}}}}
\def\vBB{{\vec {\bf B}}}

\def\CL{{\cal L}}



\def\bra#1{\left\langle#1\right|}
\def\ket#1{\left|#1\right\rangle}
\def\bbra#1{{\langle\langle}#1|}
\def\kket#1{|#1\rangle\rangle}
\def\vev#1{\left\langle{#1}\right\rangle}

\def\ketbra#1#2{ | #1 \rangle\hskip-2pt\langle #2|}



\def\be{\begin{equation}}
\def\ee{\end{equation}}
\def\({\left(}
\def\){\right)}



\newcommand{\bea}{\begin{eqnarray}}
\newcommand{\eea}{\end{eqnarray}}


\def\parfig#1#2{
\parbox{#1\textwidth}
{\includegraphics[width=#1\textwidth]{#2}}
}



%%from  Rudro Rana Biswas
\usepackage[pagebackref,  % this puts links to the page numbers where refs appear
%pdftex, 
bookmarks={false}, pdfauthor={John McGreevy}, pdftitle={Yay, physics!}]{hyperref}
\hypersetup{colorlinks=true, 
linkcolor=BrickRed, 
citecolor=Violet, 
filecolor=OliveGreen, 
urlcolor=RoyalBlue, 
filebordercolor={.8 .8 1}, 
urlbordercolor={.8 .8 0}}%http://en.wikibooks.org/wiki/LaTeX/Hyperlinks


%\usepackage{mathtools} % for inclusion arrow \xhookrightarrow{}


\renewcommand{\theequation}{\arabic{equation}}
\newif{\ifeq}           % defines a new condition @eq tested by the conditional \ifeq
\eqtrue                 % if uncommented, declares @eq to be true
%\eqfalse              % if uncommented, declares @eq to be false
%                                %
%                                % to use this, wrap text with the conditional, eg:
%                                %
%                                % \ifeq
%                                % SHOW THIS IFF \eqtrue HAS BEEN DECLARED
%                                % \fi
%
\def\answer#1
{
\ifeq
\textcolor{darkblue}{#1}
\fi
}

\begin{document}
\begin{center}

University of California at San Diego -- 
Department of Physics --
Prof.~John McGreevy

{\Large\bf  Physics 210B Non-equilibrium  Fall 2025}\\
{\Large\bf Assignment 4 \answer{--~~~ Solutions} }
\end{center}


\noindent
\hfill {\bf Due 11:59pm {Monday, October 27, 2025}} 




%I will add another problem later today.
\bigskip
\hrule




\begin{enumerate}


\item {\bf The diffusion equation in a domain.}
Let's make a crude 1d model of an elongated e-coli cell, as an interval with coordinate $x \in [-L,L]$.  
Let $c(x,t)$ be the concentration of GFP (green fluorescent protein, a genetically-attachable marker) in an e-coli cell, a function of position and time.
The initial concentration (after a strip of GFP is photobleached (removed)
across the short axis of the cell) is
\be
c\left( x,0\right) =\begin{cases}
c_{0}, & -L\leq x\leq -a \\
0, & -a\leq x\leq a \\
c_{0}, & a\leq x\leq L%
\end{cases}.
\ee
Let's find the 
profile for concentration of GFP as a function of time after photobleaching.
Assume that the concentration satisfies the diffusion equation
\[
\partial _{t}c\left( x,t\right) =D\partial _{x}^{2}c\left( x,t\right)~.
\]%
The concentration is subject to the boundary condition
\[
\partial _{x}c\left( x,t\right) =0 \text{ for }  x=\pm L
\]%
i.e., GFP cannot leak out of my e-coli.


\begin{enumerate}
\item Find the concentration $c(x,t)$ analytically.


Plot the concentration as a function of {\cor time} in units of $c_0$.


\item{} [bonus problem] Discretize the interval and solve the equation numerically.
If you're doing it numerically, you can also do it in 2d or 3d: 
put boundary conditions that GFP does not leak out of the cell (the shape of the cell is up to you), and initial conditions that a segment of width $2a$ is removed from the middle of the cell.  


A nice way to think about solving the diffusion equation numerically on a grid is that at each time step we simply take the average of the values of the neighbors of each site.   (Of course you can do better with a fixed grid size by using a better approximation to the second derivative.)





\item{} [bonus] Read \htmladdnormallink{this paper}{10.1128/JB.188.10.3442-3448.2006} that inspired this problem and write a few words about what you learned from it.

\end{enumerate}


\item {\bf Active matter.} A self-propelled particle is exactly as the name suggests, propelled by
either internal forces or external fuel source. Inanimate realizations
include diffusophoretic janus colloids and vibrated granular rods. Let us
consider a particle confined to a plane. The motion of a self-propelled
particle is typically described by a set of equations of the form%
\begin{equation*}
\partial _{t}\mathbf{r}=v_{0}\mathbf{\hat{n}}+\bm{\eta }^{T}
\end{equation*}%
\begin{equation*}
\partial _{t}\theta =\eta ^{R}
\end{equation*}%
In the above, $r$ is the center of mass of the particle and it moves along a
unit vector $\hat{n}=\cos \theta \hat{x}+\sin \theta \hat{y}$ that is pinned
to its body axis with a velocity $v_{0}$. The self propulsion is not perfect
in that it meanders a bit. This is captured by the $\theta $ equation where $
\eta ^{R}$ is a stochastic white noise that causes the direction to
fluctuate. $\eta ^{T}$ is the translational white noise that gives rise to
diffusion like we have already seen. Assume as usual that the noise correlations have zero mean,
\begin{equation*}
\left\langle \eta _{\alpha }^{T}\left( t\right) \eta _{\beta }^{T}\left(
t^{\prime }\right) \right\rangle _{T}=2D_{T}\delta _{\alpha \beta }\delta
\left( t-t^{\prime }\right)
\end{equation*}%
and%
\begin{equation*}
\left\langle \eta ^{R}\left( t\right) \eta ^{R}\left( t^{\prime }\right)
\right\rangle _{R}=D_{R}\delta \left( t-t^{\prime }\right)
\end{equation*}%
and all higher cumulants are zero, i.e., the noise is Gaussian. This is a
model for active matter that is called Active Brownian Particle (ABP). In
the following problem we want to compute the mean square dispacement $
\left\langle r_{\alpha }\left( t\right) r_{\beta }\left( t\right)
\right\rangle $.

\begin{enumerate}
\item Compute $ \vev{r_\alpha(t)} $.   


\item Show that
\begin{equation*}
\hskip-.7in
\left\langle \mathbf{\hat{n}}\left( t\right) \mathbf{\hat{n}}\left(
t^{\prime }\right) \right\rangle = A \left(
\begin{array}{cc}
\begin{array}{c}
\cos 2\theta _{0}\exp \left( -D_{R}\left[ t+t^{\prime }+2\min \left(
t,t^{\prime }\right) \right] \right)  \\
+\exp \left( -D_{R}\left[ t+t^{\prime }-2\min \left( t,t^{\prime }\right) %
\right] \right)
\end{array}
& \sin 2\theta _{0}\exp \left( -D_{R}\left[ t+t^{\prime }-2\min \left(
t,t^{\prime }\right) \right] \right)  \\
\sin 2\theta _{0}\exp \left( -D_{R}\left[ t+t^{\prime }-2\min \left(
t,t^{\prime }\right) \right] \right)  &
\begin{array}{c}
-\cos 2\theta _{0}\exp \left( -D_{R}\left[ t+t^{\prime }+2\min \left(
t,t^{\prime }\right) \right] \right)  \\
+\exp \left( -D_{R}\left[ t+t^{\prime }-2\min \left( t,t^{\prime }\right) %
\right] \right)
\end{array}%
\end{array}%
\right)
\end{equation*}
and find the constant $A$.


\item Now compute  $\left\langle r_{\alpha }\left( t\right) r_{\beta }\left(
t\right) \right\rangle $.

\item Now analyze the form of the mean square displacement at short times
i.e., $t\ll \frac{1}{D_{R}}$ and at long times, i.e., $t\gg \frac{1}{D_{R}}$ and
see what that tells you.


\item Let us set $D_{T}$ $=0$ and $\theta(0)=0$. Show that the ABP can equivalently be
represented by Langevin equations%
\begin{equation*}
\partial _{t}\mathbf{r}=\bm{Z }
\end{equation*}%
where $Z$ is a Gaussian random variable with
\begin{equation*}
\left\langle Z_{\alpha }\left( t\right) Z_{\beta }\left( t^{\prime }\right)
\right\rangle _{c}=2D_{eff}\delta _{\alpha \beta }e^{-D_{R}|t-t^{\prime }|}
\end{equation*}%
and identify $D_{eff}$ ($c$ is for `connected').
What must we take for $\vev{Z_\alpha(t)}_Z$?


\item{} [Bonus problem] Simulate a swarm of such (non-interacting) walkers in the presence of boundaries.  
Do you notice anything funny happening at the boundaries?

[Hint: keep track of the $\theta$ variable as well (which evolves in a way you understand) as the $x$ and $y$ variables.]

\item{} [Bonus problem] Suppose that the walkers interact with each other via a hard-core repulsion, i.e.~they cannot come to close.  What do you observe about the behavior of your swarm?

\end{enumerate}










\end{enumerate}
\end{document}