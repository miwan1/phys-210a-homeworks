\documentclass[12pt]{report}

\newcommand\htmladdnormallink[2]{\href{#2}{#1}}

\textheight 22cm
\textwidth 15.5cm
\oddsidemargin 0pt\evensidemargin 0pt
%\oddsidemargin 14pt\evensidemargin 0pt
%\topmargin -40pt
\topmargin-30pt
%\bottommargin0pt
\def\baselinestretch{1.1}
%\addtolength{\parskip}{1ex}
\jot=.5ex
%\parskip = 0.02in


\setlength\arraycolsep{2pt}



\usepackage{amssymb}
\usepackage{amsmath,bm}
\usepackage{amssymb}
\usepackage{graphicx}
\usepackage{amsfonts}         
\usepackage{fancybox}   

%\usepackage[numbers]{natbib}

\usepackage{enumitem}

\usepackage{slashed}




\usepackage[usenames,dvipsnames]{xcolor}%http://en.wikibooks.org/wiki/LaTeX/Colors

\definecolor{darkgreen}{rgb}{0,0.4,0}
\definecolor{darkred}{rgb}{0.4,0,0}
\definecolor{darkblue}{rgb}{0,0,0.4}
\definecolor{lightblue}{rgb}{.6,.6,0.9}
\newcommand{\cobl}{\color{darkblue}}

\newcommand{\cor}{\color{red}}
\newcommand{\cog}{\color{darkgreen}}
\newcommand{\cob}{\color{black}}

\definecolor{uglybrown}{rgb}{0.8,  0.7,  0.5}

\def\ii{{\bf i}}
\def\Ione{\mathbb{I}}
\def\UU{{\bf U}}
\def\HH{{\bf H}}
\def\pp{{\bf p}}
\def\aa{{\bf a}}
\def\qq{{\bf q}}
\def\eps{\epsilon}
\def\half{{1\over 2}}
\def\Tr{{{\rm Tr~ }}}
\def\tr{{\rm tr}}
\def\Re{{\rm Re\hskip0.1em}}
\def\Im{{\rm Im\hskip0.1em}}
\def\ppi{\boldsymbol{\pi}}
\def\pphi{\boldsymbol{\phi}}
%\def\pphi{\phi}
\def\grad{\vec \nabla}
\def\vB{\vec B}
\def\vE{\vec E}
\def\vA{\vec A}
\def\vAA{ \vec{\bf A}}
\def\vEE{{{\vec {\bf E}}}}
\def\vBB{{\vec {\bf B}}}

\def\CL{{\cal L}}



\def\bra#1{\left\langle#1\right|}
\def\ket#1{\left|#1\right\rangle}
\def\bbra#1{{\langle\langle}#1|}
\def\kket#1{|#1\rangle\rangle}
\def\vev#1{\left\langle{#1}\right\rangle}

\def\ketbra#1#2{ | #1 \rangle\hskip-2pt\langle #2|}



\def\be{\begin{equation}}
\def\ee{\end{equation}}
\def\({\left(}
\def\){\right)}



\newcommand{\bea}{\begin{eqnarray}}
\newcommand{\eea}{\end{eqnarray}}


\def\parfig#1#2{
\parbox{#1\textwidth}
{\includegraphics[width=#1\textwidth]{#2}}
}



%%from  Rudro Rana Biswas
\usepackage[pagebackref,  % this puts links to the page numbers where refs appear
%pdftex, 
bookmarks={false}, pdfauthor={John McGreevy}, pdftitle={Yay, physics!}]{hyperref}
\hypersetup{colorlinks=true, 
linkcolor=BrickRed, 
citecolor=Violet, 
filecolor=OliveGreen, 
urlcolor=RoyalBlue, 
filebordercolor={.8 .8 1}, 
urlbordercolor={.8 .8 0}}%http://en.wikibooks.org/wiki/LaTeX/Hyperlinks


%\usepackage{mathtools} % for inclusion arrow \xhookrightarrow{}


\renewcommand{\theequation}{\arabic{equation}}
\newif{\ifeq}           % defines a new condition @eq tested by the conditional \ifeq
\eqtrue                 % if uncommented, declares @eq to be true
% \eqfalse              % if uncommented, declares @eq to be false
%                                %
%                                % to use this, wrap text with the conditional, eg:
%                                %
%                                % \ifeq
%                                % SHOW THIS IFF \eqtrue HAS BEEN DECLARED
%                                % \fi
%
\def\answer#1
{
\ifeq
\textcolor{darkblue}{#1}
\fi
}

\begin{document}
\begin{center}

University of California at San Diego -- 
Department of Physics --
Prof.~John McGreevy

{\Large\bf  Physics 210B Non-equilibrium  Fall 2025}\\
{\Large\bf Assignment 4 \answer{--~~~ Solutions} }
\end{center}


\noindent
\hfill {\bf Due 11:59pm {Monday, October 27, 2025}} 




%I will add another problem later today.
\bigskip
\hrule

\begin{enumerate}
\item{\textbf{The diffusion equation in a domain}} \\
\answer{
We can use the Neumann boundary conditions:
\begin{equation}
c(x,t) = X(x)\,T(t)
\end{equation}
\begin{equation}
\partial_t c = D\,\partial_x^2 c
\quad\Rightarrow\quad
X(x)\,T'(t)=D\,X''(x)\,T(t)
\end{equation}
\begin{equation}
\frac{T'(t)}{D\,T(t)}=\frac{X''(x)}{X(x)}=-\lambda
\end{equation}
\begin{equation}
T'(t)+D\lambda\,T(t)=0,\qquad
X''(x)+\lambda\,X(x)=0
\end{equation}
\begin{equation}
X_n(x)=\cos\!\left(\frac{n\pi x}{L}\right),\quad 
\lambda_n=\left(\frac{n\pi}{L}\right)^2,\quad n=0,1,2,\dots
\end{equation}
\begin{equation}
T_n(t)=\exp\!\big(-D\lambda_n t\big)=
\begin{cases}
1,& n=0,\\[2pt]
\exp\!\!\left[-D\left(\frac{n\pi}{L}\right)^2 t\right],& n\ge1.
\end{cases}
\end{equation}
\begin{equation}
c(x,t)=A_0 X_0 + \sum_{n=1}^\infty A_n\,X_n(x)\,T_n(t)
= A_0+\sum_{n=1}^\infty A_n
\cos\!\left(\frac{n\pi x}{L}\right)
e^{-D(n\pi/L)^2 t}.
\end{equation}
Now solving the coefficients:
\begin{equation}
A_0=\frac{1}{2L}\int_{-L}^{L} f(x)\,dx,\qquad
A_n=\frac{1}{L}\int_{-L}^{L} f(x)\cos\!\left(\frac{n\pi x}{L}\right)dx
\quad(n\ge1)
\end{equation}
\begin{equation}
A_0=c_0\Big(1-\frac{a}{L}\Big)
\end{equation}
\begin{equation}
A_n
= -\,\frac{2c_0}{n\pi}\sin\!\left(\frac{n\pi a}{L}\right)
\end{equation}
Plugging into the above general solution:
\begin{equation}
c(x,t)=c_0\Big(1-\frac{a}{L}\Big)
-\;2c_0\sum_{n=1}^{\infty}
\frac{\sin\!\big(\frac{n\pi a}{L}\big)}{n\pi}\,
\cos\!\Big(\frac{n\pi x}{L}\Big)\,
\exp\!\Big[-D\Big(\frac{n\pi}{L}\Big)^2 t\Big].
\end{equation}
Therefore, to plot this solution, we observe the following:
\begin{equation}
c(x,t)\to A_0=c_0\Big(1-\frac{a}{L}\Big),
\quad \text{as } t\to\infty.
\end{equation}
\begin{equation}
c(0,t)=c_0\Big(1-\frac{a}{L}\Big)
-2c_0\sum_{n=1}^{\infty}\frac{\sin\!\big(\frac{n\pi a}{L}\big)}{n\pi}\,
e^{-D(n\pi/L)^2 t}
\end{equation}
\begin{equation}
c(\pm L,t)=c_0\Big(1-\frac{a}{L}\Big)
-2c_0\sum_{n=1}^{\infty}\frac{\sin\!\big(\frac{n\pi a}{L}\big)}{n\pi}\,
(-1)^n e^{-D(n\pi/L)^2 t}.
\end{equation}}

\item{\textbf{Active matter}} 
    \begin{enumerate}
    \item{Compute $\langle r_\alpha(t) \rangle$} 
    \answer{
    From the problem:
    \begin{equation}
    \dot{\mathbf r}(t)=v_0\,\hat{\mathbf n}(t)+\boldsymbol\eta^{T}(t), 
    \qquad
    \hat{\mathbf n}(t)=(\cos\theta(t),\,\sin\theta(t)),
    \end{equation}
    \begin{equation}
    \langle \boldsymbol\eta^{T}(t) \rangle=\mathbf 0,
    \qquad
    \dot\theta(t)=\eta^{R}(t), 
    \qquad
    \langle \eta^{R}(t)\eta^{R}(t')\rangle=D_R\,\delta(t-t').
    \end{equation}
    \begin{equation}
    \hat{\mathbf n}_0=(\cos\theta_0,\sin\theta_0).
    \end{equation}
    Compute for cosine and sine separately:
    \begin{equation}
    \langle \cos\theta(t) \rangle 
    = \cos\theta_0 \langle \cos\Delta\theta \rangle
    = \cos\theta_0\, e^{-D_R t/2},
    \end{equation}
    \begin{equation}
    \langle \sin\theta(t) \rangle
    = \sin\theta_0\, e^{-D_R t/2}.
    \end{equation}
    Need to deal with $\Delta\theta(t)$: 
    \begin{equation}
    \Delta\theta(t)\equiv \theta(t)-\theta_0
    = \int_0^t \eta^{R}(s)\,ds .
    \end{equation}
    \begin{equation}
    \langle \Delta\theta(t) \rangle
    = \int_0^t \langle \eta^{R}(s)\rangle\,ds
    = 0.
    \end{equation}
    \begin{align}
    \langle [\Delta\theta(t)]^2 \rangle
    &= \left\langle \int_0^t ds \int_0^t ds'\ \eta^{R}(s)\eta^{R}(s') \right\rangle \\
    &= \int_0^t ds \int_0^t ds'\ \langle \eta^{R}(s)\eta^{R}(s') \rangle \\
    &= \int_0^t ds \int_0^t ds'\ D_R\,\delta(s-s') \\
    &= D_R \int_0^t ds = D_R t.
    \end{align}
    Therefore, 
    \begin{equation}
    \langle \hat{\mathbf n}(t)\rangle
    =e^{-D_R t/2}\,\hat{\mathbf n}_0,
    \qquad
    \end{equation}
    \begin{equation}
    \langle \mathbf r(t)\rangle
    =v_0 \int_0^t \langle \hat{\mathbf n}(s)\rangle\,ds
    = v_0 \hat{\mathbf n}_0 \int_0^t e^{-D_R s/2}\,ds
    =\frac{2v_0}{D_R}\left(1-e^{-D_R t/2}\right)\hat{\mathbf n}_0.
    \end{equation}
    \begin{equation}
    \langle r_\alpha(t)\rangle
    =\frac{2v_0}{D_R}\left(1-e^{-D_R t/2}\right)\,n_{0,\alpha},
    \qquad \alpha\in\{x,y\}.
    \end{equation}}
    \item Show that
    \begin{equation*}
    \hskip-.7in
    \left\langle \mathbf{\hat{n}}\left( t\right) \mathbf{\hat{n}}\left(
    t^{\prime }\right) \right\rangle = A \left(
    \begin{array}{cc}
    \begin{array}{c}
    \cos 2\theta _{0}\exp \left( -D_{R}\left[ t+t^{\prime }+2\min \left(
    t,t^{\prime }\right) \right] \right)  \\
    +\exp \left( -D_{R}\left[ t+t^{\prime }-2\min \left( t,t^{\prime }\right) %
    \right] \right)
    \end{array}
    & \sin 2\theta _{0}\exp \left( -D_{R}\left[ t+t^{\prime }-2\min \left(
    t,t^{\prime }\right) \right] \right)  \\
    \sin 2\theta _{0}\exp \left( -D_{R}\left[ t+t^{\prime }-2\min \left(
    t,t^{\prime }\right) \right] \right)  &
    \begin{array}{c}
    -\cos 2\theta _{0}\exp \left( -D_{R}\left[ t+t^{\prime }+2\min \left(
    t,t^{\prime }\right) \right] \right)  \\
    +\exp \left( -D_{R}\left[ t+t^{\prime }-2\min \left( t,t^{\prime }\right) %
    \right] \right)
    \end{array}%
    \end{array}%
    \right)
    \end{equation*}
    and find the constant $A$.\\
    \answer{From part (a), we have calculated $\langle \hat{\mathbf n}(t)\rangle
    =e^{-D_R t/2}\,\hat{\mathbf n}_0,
    \qquad$. 
    \begin{equation}
    \left\langle e^{i[\theta(t)-\theta(t')]} \right\rangle
    = \exp\!\left[-\frac{D_R}{2}\left(t + t' - 2\min(t,t')\right)\right],
    \end{equation}
    \begin{equation}
    \left\langle e^{i[\theta(t)+\theta(t')]} \right\rangle
    = e^{i2\theta_0}\,
    \exp\!\left[-\frac{D_R}{2}\left(t + t' + 2\min(t,t')\right)\right].
    \end{equation}
    \begin{equation}
    A=\tfrac12
    \end{equation}
    }
    \item{Compute $\langle r_\alpha(t) r_\beta(t) \rangle$}\\ 
    \answer{
    % === Active Brownian Particle: \langle r_\alpha(t) r_\beta(t) \rangle and MSD ===
    % Equations of motion and noises
    \begin{equation}
    \dot{\mathbf r}(t)=v_0\hat{\mathbf n}(t)+\boldsymbol{\eta}^{T}(t), 
    \qquad
    \dot{\theta}(t)=\eta^{R}(t),
    \qquad
    \hat{\mathbf n}(t)=(\cos\theta,\sin\theta),
    \end{equation}
    \begin{equation}
    \langle \eta^T_\alpha(t)\eta^T_\beta(t')\rangle=2D_T\,\delta_{\alpha\beta}\delta(t-t'),
    \qquad
    \langle \eta^R(t)\eta^R(t')\rangle=D_R\,\delta(t-t').
    \end{equation}
    % Position as time integral
    \begin{equation}
    r_\alpha(t)=v_0\int_0^t n_\alpha(s)\,ds+\int_0^t \eta^T_\alpha(s)\,ds .
    \end{equation}
    % Second moment decomposition
    \begin{align}
    \langle r_\alpha(t) r_\beta(t)\rangle
    &=v_0^2\!\int_0^t\!ds\!\int_0^t\!ds'\,\langle n_\alpha(s)n_\beta(s')\rangle
    +\int_0^t\!ds\!\int_0^t\!ds'\,\langle \eta^T_\alpha(s)\eta^T_\beta(s')\rangle .
    \end{align}
    % Translational noise contribution
    \begin{equation}
    \int_0^t ds \int_0^t ds'\, 2D_T\,\delta_{\alpha\beta}\delta(s-s')
    =2D_T\,t\,\delta_{\alpha\beta}.
    \end{equation}
    % Orientation correlator
    \begin{equation}
    \langle n_\alpha(s)n_\beta(s')\rangle
    =\frac{1}{2}\,e^{-\frac{D_R}{2}|s-s'|}\,\delta_{\alpha\beta}.
    \end{equation}
    % Double-time integral of the exponential kernel
    \begin{equation}
    \int_0^t ds\int_0^t ds'\,e^{-a|s-s'|}
    =\frac{2}{a}\left[t-\frac{1-e^{-a t}}{a}\right],
    \qquad a\equiv \frac{D_R}{2}.
    \end{equation}
    % Final tensor result
    \begin{equation}
    \langle r_\alpha(t) r_\beta(t)\rangle
    =\delta_{\alpha\beta}\left[
    2D_T\,t
    +\frac{2v_0^2}{D_R}\,t
    -\frac{4v_0^2}{D_R^2}\left(1-e^{-\frac{D_R}{2}t}\right)
    \right].
    \end{equation}
    % Mean-squared displacement (trace in 2D)
    \begin{equation}
    \langle |\mathbf r(t)|^2\rangle
    =4D_T\,t+\frac{4v_0^2}{D_R}\,t
    -\frac{8v_0^2}{D_R^2}\left(1-e^{-\frac{D_R}{2}t}\right).
    \end{equation}}
    \item{Now analyze the form of the mean square displacement at short times}
    \answer{
    \paragraph{Short time $t\ll D_R^{-1}$.}
    \begin{equation}
    \langle |\mathbf r(t)|^2\rangle
    = v_0^2 t^2 + 4 D_T t + O(t^3),
    \end{equation}
    \paragraph{Long time $t\gg D_R^{-1}.$}
    Since $\(e^{-D_R t/2}\to 0\),$
    \begin{equation}
    \langle r_\alpha(t) r_\alpha(t)\rangle
    \simeq 2\left(D_T+\frac{v_0^2}{D_R}\right)t,
    \qquad
    \langle |\mathbf r(t)|^2\rangle
    \simeq 4\left(D_T+\frac{v_0^2}{D_R}\right)t,
    \end{equation}}
    \end{enumerate}
\end{enumerate}
\end{document}